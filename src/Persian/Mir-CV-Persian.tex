% A. Mir's CV in Persian

% PACKAGES AND OTHER DOCUMENT CONFIGURATIONS
\documentclass{resume} % Use the custom resume.cls style

\usepackage[left=0.75in,top=0.6in,right=0.75in,bottom=0.6in]{geometry} % Document margins
%\usepackage{datenumber, fp}
\usepackage{multicol}
\usepackage{bibentry}
\makeatletter\let\saved@bibitem\@bibitem\makeatother
%\usepackage{hyperref}

\usepackage{xepersian}

% Xepersian settings
\settextfont[Scale=1.4]{XB Niloofar}
\setlatintextfont[Scale=1]{Times New Roman}
\setdigitfont[Scale=1.2]{XB Niloofar}

% Definitions
\renewcommand\labelitemi{-}
%\newcounter{birth}%
%\newcounter{now}%
%
%\setmydatenumber{birth}{1993}{04}{07}%
%\setmydatenumber{now}{\the\year}{\the\month}{\the\day}%
%\FPsub\result{\thenow}{\thebirth}
%\FPdiv\myage{\result}{365.2425}

%\newcommand{\perspaper}{[\textbf{in Persian}]}


\name{امیرمحمود میر} % Your name
%\address{123 Broadway \\ City, State 12345} % Your address
%\address{123 Pleasant Lane \\ City, State 12345} % Your secondary addess (optional)
%\address{(000)~$\cdot$~111~$\cdot$~1111 \\ john@smith.com} % Your phone number and email

\begin{document}
\nobibliography{ref}
\bibliographystyle{ieeetr-fa}


\begin{rSection}{اطلاعات تماس}
\begin{multicols}{2}
%Age: \FPtrunc\myage{\myage}{0}\myage\\
ایمیل: \lr{mir-am@hotmail.com}\\
پروفایل \lr{Linkedin}: \lr{linkedin.com/in/mir93} \\
پروفایل \lr{GitHub}: \lr{github.com/mir-am} \\

\vfill\null
\columnbreak
موبایل: 09361819828 \\
وب‌سایت شخصی: \lr{www.mirblog.me}\\
%
\end{multicols}
\end{rSection}

%----------------------------------------------------------------------------------------
%	EDUCATION SECTION
%----------------------------------------------------------------------------------------

\begin{rSection}{سوابق تحصیلی}

{\textbf{دانشگاه آزاد اسلامی - واحد تهران شمال}} \hfill {بهمن ۹۴ - بهمن ۹۷} \\ 
کارشناسی ارشد - مهندسی کامپیوتر \\
گرایش هوش مصنوعی \smallskip \\
عنوان پایان‌نامه: مقاوم‌سازی ماشین بردار پشتیبان دو قلو در برابر داده‌های نویزی \\
معدل کل: $17.06$ \\
دروس گذرانده: مبانی هوش مصنوعی، یادگیری ماشین، شناسایی آماری الگو، رایانش تکاملی، شبکه‌های عصبی، پردازش تصویر، بینایی کامپیوتر، پردازش زبان طبیعی، تئوری بازی‌ها، روش تحقیق \\


%Minor in Artificial Intelligence \& Machine Learning  \smallskip \\
%Thesis subject: Robust Twin Support Vector Machine for Noisy Data \\
%Overall GPA: 3.41 out of 4 \\
%Courses: Introduction to Artificial Intelligence, Machine Learning, Statistical Pattern Recognition, Evolutionary Computation, Image Processing, Computer Vision, Natural Language Processing, Game Theory, Research Methodology \\
%
{\textbf{دانشگاه تهران}} \hfill {مهر ۹۰ - تیر ۹۴} \\
کارشناسی \\
پروژه نهایی: کاربرد زبان برنامه‌نویسی پایتون در آب ‌و ‌هواشناسی \\
دروس گذارنده: ریاضیات گسسته، ساختمان داده، طراحی الگوریتم، مدار منطقی، زبان اسمبلی، سیستم عامل‌ها، معماری کامپیوتر، پایگاه‌داده، نظریه زبان‌ها و ماشین‌ها، طراحی و پیاده‌سازی زبان‌های برنامه‌نویسی \\
%B.Sc degree \\
%Final project: Applications of Python Programming Language in Climatology \\ 
%Courses: Discrete Mathematics, Data Structures, Algorithm Design, Digital Logic, Assembly Language, Operating Systems, Computer Architecture, Database Systems, Formal Languages and Automata, Design of Programming Languages
\end{rSection}
%
%%----------------------------------------------------------------------------------------
%%	WORK EXPERIENCE SECTION
%%----------------------------------------------------------------------------------------

\begin{rSection}{سوابق‌ کاری}
\def\jobtitle{دستیار پژوهشی در آزمایشگاه یادگیری ماشین و متن کاوی}
\def\employername{پژوهشگاه علوم و فناوری اطلاعات ایران (ایرانداک)}
\def\years{تیر ۹۶ - اکنون}
\def\location{تهران، ایران}

\begin{rSubsection}{\employername}{\years}{\jobtitle}{\location}

\textbf{فعالیت و دستاوردها:}
\item چاپ یک مقاله \lr{ISI}، یک مقاله علمی - پژوهشی و سه مقاله کنفرانسی (\lr{ISC}) در زمینه‌های یادگیری ماشین، پردازش زبان طبیعی و آموزش الکترونیکی
\item طراحی و پیاده‌سازی الگوریتم‌های یادگیری ماشین در زبان \lr{Python} و \lr{C++}
\item توسعه نرم‌افزار \lr{LightTwinSVM} برای حل کردن مسائل دسته‌بندی توسط پژوهشگران و دانشجویان

%%\item Designed and implemented machine learning algorithms in C++ and Python.
%%\item Published a refereed machine learning research paper in the Journal of Applied Intelligence.
%%\item Developed LightTwinSVM program for the research and classification tasks.
\end{rSubsection}

%%------------------------------------------------
%
\end{rSection}
%
%% Publication
\begin{rSection}{سوابق علمی و پژوهشی}
\small\textbf{مقالات چاپ شده در مجلات \lr{ISI}}
\begin{latin}
\begin{itemize}
	\item \bibentry{mirjul2018}
	%\item \bibentry{mirnov2018} \perspaper
\end{itemize}
\end{latin}

\small\textbf{مقالات چاپ شده در مجلات علمی-پژوهشی}
\begin{itemize}
	\item \bibentry{mirnov2018}
\end{itemize}

\small\textbf{مقالات ارائه شده در کنفرانس‌های \lr{ISC}}
\begin{itemize}
	\item \bibentry{mirjul2017}
	\item \bibentry{mirsep2017}
	\item \bibentry{mirfeb2018}
\end{itemize}
\end{rSection}

\begin{rSection}{پروژه‌ها}
	
	{\textbf{نرم‌افزار \lr{LightTwinSVM}} } \hfill {\lr{https://github.com/mir-am/LightTwinSVM}} \\ 
	پیاده‌سازی ساده و سریع دسته‌بند ماشین بردار پشتیبان دوقلو (\lr{TwinSVM})  \\
	%\textbf{Features:}
	\begin{itemize}
		\item خط فرمان ساده برای اجرای دسته‌بند \lr{TwinSVM}
		\item بهبود و پیاده‌سازی الگوریتم \lr{clipDCD} در \lr{C++} برای حل کردن مسائل بهینه‌سازی دسته‌بند \lr{TwinSVM}
		\item پشتیبانی از توابع هسته خطی، \lr{RBF} و مستطیلی
		\item پشتیبانی از دسته‌بندی دوکلاسه و روش‌های چندکلاسه \lr{One-vs-All} و \lr{One-vs-One}
		\item قابلیت جستجوی شبکه‌ای برای پیدا کردن مقادیر بهینه پارامترهای دسته‌بند
        \item ذخیره‌سازی نتایج کامل دسته‌بندی در یک فایل \lr{Excel}	
	%\item A simple console program for running TwinSVM classifier
%	\item The clipDCD algorithm was improved and implemented in C++ for solving optimization problems of TwinSVM.
%	\item Linear, RBF kernel and Rectangular are supported.
%	\item Binary and Multi-class classification (One-vs-All \& One-vs-One) are supported.
%	\item It supports grid search over C and gamma parameters.
%	\item Detailed classification result will be saved in a spreadsheet file.
	\end{itemize}
	
\end{rSection}

%%----------------------------------------------------------------------------------------
%%	TECHNICAL STRENGTHS SECTION
%%----------------------------------------------------------------------------------------

\begin{rSection}{مهارت‌های فنی}
	
	\begin{tabular}{ @{} >{\bfseries}r @{\hspace{6ex}} l }
		زبان‌های برنامه‌نویسی & \lr{Python, C, Modern C++} \\
		سیستم‌عامل‌ها & \lr{Linux(Ubuntu), Windows} \\
		پایگاه داده & \lr{MySQL, Microsoft SQL} \\
		سیستم \lr{ٰVersion Control} & \lr{Git, GitHub} \\
		حروف‌چینی & \small{\LaTeX} \\
	
%		Operating Systems & Linux (Ubuntu), Windows  \\
%		Databases & MySQL, Microsoft SQL \\
%		Source Control & Git, GitHub \\
%		Typesetting & LaTeX
	\end{tabular}
\end{rSection}
%
%% Research Interests
\begin{rSection}{زمینه‌های پژوهش}
	
	\begin{itemize}
		\item یادگیری ماشین
		\item دسته‌بندی الگو‌ها
		\item پردازش زبان طبیعی
	\end{itemize}
	
\end{rSection}
%
%% Language test scores
%\begin{rSection}{Languages}
%	
%	\begin{itemize}
%		\item English
%		\item Persian 
%	\end{itemize}
%	
%\end{rSection}
%
%
%
%%----------------------------------------------------------------------------------------
%%	EXAMPLE SECTION
%%----------------------------------------------------------------------------------------
%
%%\begin{rSection}{Section Name}
%
%%Section content\ldots
%
%%\end{rSection}
%
%%----------------------------------------------------------------------------------------

\end{document}
