%%%%%%%%%%%%%%%%%%%%%%%%%%%%%%%%%%%%%%%%%
% Medium Length Professional CV
% LaTeX Template
% Version 2.0 (8/5/13)
%
% This template has been downloaded from:
% http://www.LaTeXTemplates.com
%
% Original author:
% Trey Hunner (http://www.treyhunner.com/)
%
% Important note:
% This template requires the resume.cls file to be in the same directory as the
% .tex file. The resume.cls file provides the resume style used for structuring the
% document.
%
%%%%%%%%%%%%%%%%%%%%%%%%%%%%%%%%%%%%%%%%%

%----------------------------------------------------------------------------------------
%	PACKAGES AND OTHER DOCUMENT CONFIGURATIONS
%----------------------------------------------------------------------------------------

\documentclass{resume} % Use the custom resume.cls style

\usepackage[left=0.75in,top=0.6in,right=0.75in,bottom=0.6in]{geometry} % Document margins
\usepackage{datenumber, fp}
\usepackage{multicol}
\usepackage{lastpage}
\usepackage{fancyhdr}
\usepackage{paracol} % For creating two columns environment
\usepackage{bibentry} % For inserting bib items in CV
%\makeatletter\let\saved@bibitem\@bibitem\makeatother
%\usepackage{hyperref}

% Definitions
\newcounter{birth}%
\newcounter{now}%

\setmydatenumber{birth}{1993}{04}{07}%
\setmydatenumber{now}{\the\year}{\the\month}{\the\day}%
\FPsub\result{\thenow}{\thebirth}
\FPdiv\myage{\result}{365.2425}

\fancyhf{} % sets both header and footer to nothing
\renewcommand{\headrulewidth}{0pt}

% Page Style
\pagestyle{fancy}
\cfoot{\thepage\ of \pageref{LastPage}}

\newcommand{\perspaper}{[\textbf{in Persian}]}

\name{Amir M. Mir} % Your name
\address{Delft, The Netherlands} % Your address
%\address{123 Pleasant Lane \\ City, State 12345} % Your secondary addess (optional)
%\address{(000)~$\cdot$~111~$\cdot$~1111 \\ john@smith.com} % Your phone number and email

\begin{document}
\nobibliography{ref}
\bibliographystyle{plain}

\begin{rSection}{Contact Detail}

\begin{multicols}{2}
%Age: \FPtrunc\myage{\myage}{0}\myage\\
E-mail: s.a.m.mir@tudelft.nl \\
Linkedin profile: linkedin.com/in/mir93 \\
GitHub profile: github.com/mir-am \\
\vfill\null
\columnbreak
Mobile: +31626303013\\
Website: www.mirblog.net\\
\end{multicols}
~\\[-1.5cm]
Google Scholar profile: scholar.google.com/citations?user=IZB4GI8AAAAJ\&hl
%\begin{center}
%\begin{minipage}[t]{0.50\textwidth}
	%\raggedright
	%First Name:
	
	
%\end{minipage}%
%\begin{minipage}[t]{0.50\textwidth}
	%\raggedleft
	%Middle Name:  \\

	
%\end{minipage}
%\begin{minipage}[t]{0.30\textwidth}
%	\raggedleft
%	%Last Name: Mir \\
%\end{minipage}%	

%\end{center}	


	
\end{rSection}

%----------------------------------------------------------------------------------------
%	EDUCATION SECTION
%----------------------------------------------------------------------------------------

\begin{rSection}{Education}
{\bf Delft University of Technology} \hfill {Oct. 2019 - Present} \\
PhD Candidate in Software Engineering \\
At the Software Analytics lab, I do research at the intersection of Machine Leaning and Programming Languages.

{\bf Islamic Azad University} (North Tehran Branch) \hfill {Feb. 2016 - Jan. 2019} \\ 
M.Sc in Computer Engineering  \\
Specialization in Artificial Intelligence \& Machine Learning  \smallskip \\
Thesis subject: Robust Twin Support Vector Machine for Noisy Data \hfill \textbf{Thesis grade:} 4/4 (A+)  \\
Overall \textbf{GPA}: 3.52 out of 4 \\
Courses: Introduction to Artificial Intelligence, Artificial Neural Networks, Machine Learning, Statistical Pattern Recognition, Evolutionary Computation, Image Processing, Computer Vision, Natural Language Processing, Game Theory, Research Methodology \\

{\bf University of Tehran} \hfill {Oct. 2011 - Jul. 2015} \\
B.Sc degree \\
Final project: Applications of Python Programming Language in Climatology \\ 
Courses: Discrete Mathematics, Data Structures, Algorithm Design, Digital Logic, Assembly Language, Operating Systems, Computer Architecture, Database Systems, Formal Languages and Automata, Design of Programming Languages
\end{rSection}

%----------------------------------------------------------------------------------------
%	WORK EXPERIENCE SECTION
%----------------------------------------------------------------------------------------

\begin{rSection}{Work Experience}

\begin{rSubsection}{Iranian Research Institute for Information Science and Technology }{Jul. 2017 - Sep. 2019}{Research Assistant at Machine Learning and Text Mining Lab }{Tehran, Iran}
	
\textbf{Achievements and Contributions}
\item Published two research papers in scholarly journals.
\item Presented three research papers at international and national conferences.
%\item Published a refereed machine learning research paper in the Journal of Applied %Intelligence.
\item Designed and implemented machine learning algorithms in C++ and Python.
\item Developed LightTwinSVM program and LIBTwinSVM library for the research and classification tasks.
\item Taught students to do research and solve problems.
\end{rSubsection}

%------------------------------------------------

\end{rSection}

\newpage
% Publication
\begin{rSection}{Publication}
\subsection*{Journals}
\begin{itemize}
	\item \bibentry{Nasiri2020}
	\item \bibentry{mirjul2018}
	\item \bibentry{mir2019joss}
	\item \bibentry{mir2020libtwinsvm}
	\item \bibentry{mirnov2018} \perspaper
\end{itemize}

\subsection*{Conferences}
\begin{itemize}
	\item \bibentry{mirfeb2018} \perspaper
	\item \bibentry{mirsep2017} \perspaper
	\item \bibentry{mirjul2017} ] \perspaper
\end{itemize}	
\end{rSection}

\begin{rSection}{Talks}
\begin{itemize}
	\item FASTEN: Intelligent Software Package Management, OW2con'2020, online (Jun. 2020)
	\item Deep Learning Type Inference for Dynamic Programming Languages, SERG Lunch, TU Delft, The Netherlands (Apr. 2020)
	\item LIBTwinSVM: A Library for Twin Support Vector Machine, SERG Lunch, TU Delft, The Netherlands (Nov. 2019)
\end{itemize}
\end{rSection}

\begin{rSection}{Software Projects}
	
	{\bf FASTEN} \hfill {https://www.fasten-project.eu/} \\ 
	Fine-Grained Analysis of Software Ecosystems as Networks \\
	
	%I have worked as a software developer in the FASTEN project, which aims to make software package management systems more intelligent and robust.
	
    \begin{itemize}
		\item Wrote a report on specifications of plug-ins for FASTEN services.
		\item Designed and developed dataflow plug-ins using Apache Kafka.
		\item Developed the FASTEN server which is a lightweight run time environment.
		\item Deployed the FASTEN plug-ins and services on Kubernetes clusters.
		\item Developed a web crawler for scraping coordinates of Maven packages.
	\end{itemize}
	
	\newpage
	{\bf LIBTwinSVM} \hfill {https://github.com/mir-am/LIBTwinSVM} \\ 
	A Library for Twin Support Vector Machines \\
	
	\begin{itemize}
		\item A simple and user-friendly Graphical User Interface
		\item Highly optimized implementation of standard TwinSVM and Least Squares TwinSVM classifiers.
		\item A Python application programming interface for employing TwinSVM estimators.
		\item A feature-rich visualization tool to show decision boundaries and geometrical interpretation of TwinSVMs.
		\item The best-fitted classifier can be saved on the disk.
	\end{itemize}
	
	{\bf LightTwinSVM} \hfill { https://github.com/mir-am/LightTwinSVM} \\ 
	A simple and fast implementation of standard TwinSVM classifer  \\
	
	%\textbf{Features:}
	\begin{itemize}
	\item A simple console program for running TwinSVM classifier
	\item The clipDCD algorithm was improved and implemented in C++ for solving optimization problems of TwinSVM.
	\item Linear, RBF kernel and Rectangular are supported.
	\item Binary and Multi-class classification (One-vs-All \& One-vs-One) are supported.
	\item It supports grid search over $C$ and gamma parameters.
	\item Detailed classification result will be saved in a spreadsheet file.
	\item Used continuous integration services (Travis CI \& AppVeyor) to build and test the program on Linux, OSX, and Windows systems.
	\end{itemize}

\end{rSection}

%----------------------------------------------------------------------------------------
%	TECHNICAL STRENGTHS SECTION
%----------------------------------------------------------------------------------------

\begin{rSection}{Technical Skills}
	
	\begin{tabular}{ @{} >{\bfseries}l @{\hspace{6ex}} l }
		Research & Planning, Data Collection, Evaluating Sources, \\
		& Critical Thinking, Documenting and Reporting \\
		Programming Languages & Python, C, Modern C++\\
		Software Development &Life Cycle, Clean Code, Debugging, Documentation, \\
		& Continuous Integration, Unit Testing, Profiling, Maintenance \\
		Machine Learning Libraries & Scikit-learn, PyTorch, Keras, TensorFlow, mlpack \\
		Operating Systems & MacOS, Linux (Ubuntu), Windows  \\
		Databases & MySQL, Microsoft SQL \\
		Source Control & Git, GitHub \\
		Typesetting & LaTeX
	\end{tabular}
	
\end{rSection}

\begin{paracol}{2}
\begin{rSection}{Research Interests}
	
	\begin{itemize}
		\item Software Engineering
		\item Machine Learning
		\item Pattern Classification
		\item Natural Language Processing
	\end{itemize}
	
\end{rSection}
\switchcolumn
% Language test scores
\begin{rSection}{Languages}
	
	\begin{itemize}
		\item English
		\item Persian 
	\end{itemize}
	
\end{rSection}
\end{paracol}

\begin{rSection}{Certificates}
	
	\begin{itemize}
		\item \textbf{International School on Software Engineering (ISE 2020)}, the Free University of Bozen-Bolzano
         and the University of Innsbruck - Jul. 2020
		\item \textbf{Unix Tools: Data, Software and Production Engineering}, edX - Jun. 2020
		\item \textbf{Software Development Fundamentals}, Microsoft Virtual Academy - Mar. 2015
		\item \textbf{Introduction to Programming with Python}, Microsoft Virtual Academy - Mar. 2015
	\end{itemize}
	
\end{rSection}

\begin{rSection}{Services}
	
	\begin{itemize}
		\item \textbf{ICSE 2020}, Member of Virtualization and Live Streaming Team - Jul. 2020
		\item \textbf{ICSE 2021}, Sub-reviewer
		\item \textbf{FSE 2020}, Sub-reviewer
	\end{itemize}
	
\end{rSection}

%----------------------------------------------------------------------------------------
%	EXAMPLE SECTION
%----------------------------------------------------------------------------------------

%\begin{rSection}{Section Name}

%Section content\ldots

%\end{rSection}

%----------------------------------------------------------------------------------------

\end{document}
